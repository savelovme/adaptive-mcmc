
\section{Martingales}
See \cite{kallenberg2021foundations} 
% +(Appendix E of \cite{douc2018markov}?)

\section{Markov processes and transition kernels}
\subsection*{Notation}

Given a measurable space $(\mathcal{X}, \mathcal{F}_{\mathcal{X}})$, let $\mathbb{M}_1(\mathcal{X})$ denote the set of all probability measures on it.
Let $P$ be a transition kernel on $(\mathcal{X}, \mathcal{F}_{\mathcal{X}})$ which defines a linear operator $P \colon \mathbb{M}_1(\mathcal{X}) \to \mathbb{M}_1(\mathcal{X})$ given by
\[
\mu P(B) = \int_{\mathcal{X}} P(x, B) \mu (dx) , \quad \mu \in \mathbb{M}_1(\mathcal{X}), B  \in  \mathcal{F}_{\mathcal{X}} .
\]
Further, for a measurable function $f \colon \mathcal{X} \to \mathbb{R}$ and $\mu \in \mathbb{M}_1(\mathcal{X})$ we have 
\[
\mu(f) = \int_{\mathcal{X}} f(x) \mu(dx),
\]
\[ 
Pf(x) = \int_{\mathcal{X}} f(y) P(x, dy) 
\]
whenever well-defined.

\section{Wasserstein ergodicity}

See \cite{douc2018markov} and \cite{rudolf2017perturbationtheorymarkovchains}

Let $d$ be a metric 
% (possibly different from the one which makes $\mathcal{X}$ Polish) 
which is assumed to be lower semi-continuous with respect to the product topology of $\mathcal{X}$. For two probability measures $\nu, \mu \in \mathbb{M}_1(\mathcal{X})$ we define the Wasserstein distance by
\[
\mathcal{W}(\nu, \mu):=\inf_{\xi \in \mathcal{C}(\nu, \mu)} \int_{\mathcal{X}^2} d(x,y) \xi(dx , dy) ,
\]
where $\mathcal{C}(\mu_1, \mu_2)$ is the set of all couplings of $\nu, \mu$, that is, all probability measures on $\mathcal{X} \times \mathcal{X}$ with marginals $\nu$ and $\mu$.

For a measurable function $f \colon \mathcal{X} \to \mathbb{R}$ we denote Lipschitz semi-norm as
\[
\| f \|_d = \sup_{\substack{x, y \in \mathcal{X} \\ x \neq y}} \frac{\left\vert f(x) - f(y) \right\vert}{d(x, y)}
\]
% We write $\mathrm{Lip}_1$ for the set of all functions that are Lipschitz (w.r.t. $d$) with constant $1$. 

\begin{theorem}[Kantorovich-Rubenstein duality]
\label{thm:kantorovich_rubenstein_duality}
\[
\mathcal{W}(\nu, \mu) = \sup_{ \| f \|_d \leq 1} \left\vert \nu(f) - \mu(f) \right\vert.
\]
\end{theorem}

Further, we define the following quantities: 
\begin{itemize}
\item[1)] The eccentricity (see \cite{joulin2010curvature}) is defined as 
\[
E(x) = \int_\mathcal{X} d(x, y) \pi(dy).
\] 
\item[2)] Coarse diffusion coefficient (see \cite{joulin2010curvature})
\[
\mathrm{diff}(x, \gamma) = \int_\mathcal{X} \int_\mathcal{X} d(x', x'')^2 P_{\gamma}(x, dx') P_{\gamma}(x, dx'').
\]
\item[3)] The Wasserstein contraction coefficient  for transition kernel $P$ is 
\[
\tau(P) = \sup_{\substack{x, y \in \mathcal{X} \\ x \neq y}} \frac{\mathcal{W} (\delta_x P, \delta_y P)}{d(x, y)}.
\]
\end{itemize}

\begin{proposition}
 For the Wasserstein contraction coefficient one has the properties of
\begin{itemize}
\item[i)] submultiplicativity, that is $\tau(P \tilde{P} ) \leq \tau(P) \tau(\tilde{P})$, and,
\item[ii)] contractivity, that is $\mathcal{W}(\nu P , \mu P) \leq \tau(P) \mathcal{W}(\nu, \mu)$,
\end{itemize}
for any transition kernels $P$, $\tilde{P}$ and any probability measures $\nu, \mu \in \mathbb{M}_1(\mathcal{X})$.  
\end{proposition}